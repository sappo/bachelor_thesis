% Sprache
\usepackage[english,ngerman]{babel} % englische und deutsche Rechtschreibung
\usepackage[utf8]{inputenc} % Unicode Text 
\usepackage[T1]{fontenc} % Umlaute und deutsches Trennen
\usepackage{textcomp} % Euro
\usepackage[hyphens]{url}
% statt immer Ab\-schluss\-ar\-beit zu schreiben
% einfach hier sammeln mit -. 
\hyphenation{Ab-schluss-ar-beit}
% Vorsicht bei Umlauten und Bindestrichen
\hyphenation{Ver-st\"ar-ker-aus-gang}
 % eigene Hyphenations, die für das Dokument gelten
\usepackage{amssymb} % Symbole
\usepackage{enumitem}
\usepackage{tabularx}

%% Fonts, ein kompletter Satz an Optionen
% Times New Roman, gewohnter Font mit ok tt und serifenlos
%\usepackage{mathptmx} 
%\usepackage[scaled=.95]{helvet}
%\usepackage{courier}
% Palatino, mal was anderes, auch mit ok tt und serifenlos
% empfohlen
\usepackage{mathpazo} % Palatino, mal was anderes
\usepackage[scaled=.95]{helvet}
\usepackage{courier}
% New Century Schoolbook sieht auch nett aus (macht auch tt und serifenlos)
%\usepackage{newcent}

% zusätzlich: Default serifenlos mit Helvetica 
% ich kann es nicht mehr sehen...
%\renewcommand{\familydefault}{\sfdefault}

\usepackage{microtype}

% Bilder und Listings
\usepackage{graphicx} % wir wollen Bilder einfügen
\usepackage{subfig} % Teilbilder
\usepackage{wrapfig} % vielleicht doch besser vermeiden
\usepackage{listings} % schöne Quellcode-Listings
% ein paar Einstellungen für akzeptable Listings
\lstset{basicstyle=\sffamily, columns=[l]flexible, mathescape=true, showstringspaces=false, numbers=left, numberstyle=\tiny}
\lstset{language=python} % und nur schöne Programmiersprachen ;-)
% und eine eigene Umgebung für Listings
\usepackage{float}
\newfloat{listing}{htbp}{scl}[chapter]
\floatname{listing}{Listing}

% Seitenlayout
\usepackage[paper=a4paper,width=14cm,left=35mm,height=22cm]{geometry}
%\usepackage[paper=a4paper,width=14cm,height=22cm]{geometry}
\usepackage{setspace}
\linespread{1.15}
\setlength{\parskip}{0.5em}
\setlength{\parindent}{0em} % im Deutschen Einrückung nicht üblich, leider
\setlength{\marginparwidth}{0pt}

% Seitenmarkierungen 
\newcommand{\phv}{\fontfamily{phv}\fontseries{m}\fontsize{9}{11}\selectfont}
\usepackage{fancyhdr} % Schickere Header und Footer

\pagestyle{fancy}
\renewcommand{\chaptermark}[1]{\markboth{#1}{}}
\renewcommand{\sectionmark}[1]{\markright{#1}}
\fancyhead[LO]{\phv \nouppercase{\leftmark}}
\fancyhead[RE]{\phv \nouppercase{\rightmark}}
\fancyhead[RO,LE]{\phv \thepage}
\fancyfoot[C]{\ } % Seitenzahl unten nur Kapitel


% Theorem-Umgebungen
\newtheorem{definition}{Definition}[chapter]
\newtheorem{satz}{Satz}[chapter]
\newtheorem{lemma}[satz]{Lemma} % gleicher Zähler wie Satz
\newtheorem{theorem}{Theorem}[chapter]
\newenvironment{beweis}[1][Beweis]{\begin{trivlist}
\item[\hskip \labelsep {\textit{#1 }}]}{\end{trivlist}}
\newcommand{\qed}{\hfill \ensuremath{\square}}

% Inhaltsverzeichnis
\setcounter{tocdepth}{1}
\setcounter{secnumdepth}{2}

% Quellen teilen
\usepackage{bibtopic} 

% Hochschule Logo, noch nicht perfekt
\usepackage{hsrmlogo}

% Spezialpakete
\usepackage{epigraph}
\setlength{\epigraphrule}{0pt} % kein Trennstrich

% damit wir nicht so viel tippen müssen, nur für Demo 
\usepackage{blindtext} 

% Zum Zeigen von Fehlern
\usepackage{soul}
\newcommand*\falsch{\st}

% Links im PDF
\usepackage{hyperref}
\hypersetup{
    colorlinks=true,
    citecolor=black,
    filecolor=black,
    linkcolor=black,
    urlcolor=black
}

% Kommentare
\usepackage{comment}

\usepackage{amsmath}

\renewcommand{\lstlistingname}{Auflistung}

\usepackage{color}
\usepackage{xcolor}
\usepackage{caption}

\usepackage{calc} 
\newlength\tdima \newlength\tdimb \setlength\tdima{ \fboxsep+\fboxrule} \setlength\tdimb{-\fboxsep+\fboxrule}

\DeclareCaptionFont{white}{\color{white}}
\DeclareCaptionFormat{listing}{%
  \parbox{\textwidth}{\colorbox{gray}{\parbox{\textwidth}{#1#2#3}}\vskip-4pt}}
\captionsetup[lstlisting]{format=listing,labelfont=white,textfont=white}
\lstset{frame=lrb,rulecolor= \color{gray},xleftmargin=\tdima,xrightmargin=\tdimb}

\usepackage{wasysym} % für die benutzten Symbole
\usepackage{ulem} % durchstreichen